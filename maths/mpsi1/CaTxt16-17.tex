\documentclass[12pt,a4paper]{article}

\textheight=25cm
\topmargin=-50pt
\input{/home/skanderk/.latex/intro2.sty}

\begin{document}

\begin{center}
\Large\bf CAHIER DE TEXTES DE MATHÉMATIQUES\\
MPSI 1 La Martinière Monplaisir\\ 2016-2017
\end{center}
\vspace{1cm}
\vspace{.4cm}

% 
% \noindent\textbf{\bf Jeudi 18 juin 2016 } \\
% 
% 
% \noindent\textbf{\bf Mercredi 17 juin 2016 } \\
% 
% \noindent\textbf{\bf Lundi 15 juin 2016 } \\
% \bu\ Interrogation n° 24.\\
% \bu\ Cours : \bf Chapitre XXV \rm : Dénombrement : fin.\\
% \bu\ Exercices : feuille n° 24, ex. 20 à 22 et feuille n° 25, ex. 2, 3, 6 et 7.\vspace{.4cm}\\
% 
% \noindent\textbf{\bf Vendredi 12 juin 2016 } \\
% \bu\ Exercices : feuille n° 24, ex. 17 à 19.\vspace{.4cm}\\
% 
% \noindent\textbf{\bf Jeudi 11 juin 2016 } \\
% \bu\ Exercices : feuille n° 24, ex. 12 à 16.\vspace{.4cm}\\
% 
% \noindent\textbf{\bf Mercredi 10 juin 2016 } \\
% \bu\ Cours : 3 - Automorphismes orthogonaux (fin).\\
% \bu\ Exercices : feuille n° 24, ex. 6 à 11.\vspace{.4cm}\\
% 
% \noindent\textbf{\bf Lundi 08 juin 2016 } \\
% \bu\ Interrogation n° 23.\\
% \bu\ Cours : 3 - Automorphismes orthogonaux (suite).\\
% \bu\ Exercices : feuille n° 23, ex. 14 et feuille n° 24, ex. 1 à 5.\vspace{.4cm}\\
% 
% \noindent\textbf{Vendredi 05 juin 2016 }\\ 

% 
% \noindent\textbf{ Mercredi 29 avril 2016 } \\
% \bu\ Cours : 4 - Séries absolument convergentes ; 5 - Représentation décimale des réels ; 6 - 
% Compléments.\\
% \bu\ Exercices : feuille n° 20, ex. 19, et feuille n° 21, ex. 1.\vspace{.4cm}\\
%

%\bu\ Exercices : feuille n° 20, ex. 11 et 13 à 18.\vspace{.4cm}\\

% \noindent\textbf{\bf Lundi 13 juin 2017 }\\
% \bu\ Cours : 1 - Prolégomènes ; 2 - Séries à termes réels positifs ; 3 - Comparaison série - intégrale.\\
% \bu\ Exercices : feuille n° 25, ex. 14 à 22.\vspace{.4cm}\\
% 
% \noindent\textbf{Vendredi 10 juin 2017 }\\
% \bu\ Devoir surveillé n° 10.\\
% \bu\ Cours : 3 - Automorphismes orthogonaux (fin).\\
% \bu\ Exercices : feuille n° 25, ex. 12 et 13.\vspace{.4cm}\\
% 
% \noindent\textbf{\bf Jeudi 09 juin 2017 }\\
% \bu\ Cours : 3 - Automorphismes orthogonaux (suite).\\
% \bu\ Exercices : feuille n° 25, ex. 8 à 11.\vspace{.4cm}\\
% 
% \noindent\textbf{\bf Lundi 06 juin 2017 } \\
% \bu\ Interrogation n° 22.\\
% \bu\ Cours : 3 - Automorphismes orthogonaux (début).\\
% \bu\ Exercices : feuille n° 25, ex. 1 à 7.\vspace{.4cm}\\
% 
% \noindent\textbf{Vendredi 03 juin 2017 }\\
% \bu\ Cours : 2 - Orthogonalité (fin).\\
% \bu\ Exercices : feuille n° 24, ex. 16.\vspace{.4cm}\\
% 
% \noindent\textbf{\bf Jeudi 02 juin 2017 }\\
% $\bullet$\ Distribution : DM n° 21 (à rendre le 09 juin).\\
% \bu\ Cours :  2 - Orthogonalité (début).\vspace{.4cm}\\
% 
% \noindent\textbf{\bf Mercredi 01 juin 2017 }\\
% \bu\ Cours : 1 - Produits scalaires, normes et distances (début).\\
% \bu\ Exercices : feuille n° 24, ex. 13 et 14 (début).\vspace{.4cm}\\
% 
% \noindent\textbf{\bf Lundi 30 mai 2017 }\\
% \bu\ Interrogation n° 21.\\
% \bu\ Cours : 5 - Déterminant d'une matrice carrée (fin).\\
% \bf Chapitre XXIV \rm : Espaces vectoriels euclidiens : 1 - Produits scalaires, normes et distances 
% (début).\\
% \bu\ Exercices : feuille n° 24, ex. 7 à 12.\vspace{.4cm}\\
% 
% \noindent\textbf{Vendredi 27 mai 2017 }\\
% \bu\ Cours : 5 - Déterminant d'une matrice carrée (début).\vspace{.4cm}\\
% 
% \noindent\textbf{\bf Jeudi 26 mai 2017 } \\
% $\bullet$\ Distribution : DM n° 21 (à rendre le 02 juin).\\
% \bu\ Cours :  4 - Déterminant d'un endomorphisme.\\
% \bu\ Exercices : feuille n° 24, ex. 5 et 6.\vspace{.4cm}\\
% 
% \noindent\textbf{\bf Mercredi 25 mai 2017 } \\
% \bu\ Cours : 3. Déterminant d’une famille de vecteurs.\\
% \bu\ Exercices : feuille n° 24, ex. 1 à 4.\vspace{.4cm}\\
% 
% \noindent\textbf{\bf Lundi 23 mai 2017 } \\
% \bu\ Interrogation n° 20.\\
% \bu\ Cours : 2. Applications multilinéaires (fin).\\
% \bu\ Exercices : feuille n° 23, ex. 12, 14, 15, 16 et 18 à 20.\\
% $\bullet$\ À faire pour mercredi 25/05 : feuille n° 23, ex. 17.\vspace{.4cm}\\
% 
% \noindent\textbf{Vendredi 20 mai 2017 }\\
% \bu\ Cours : 1 - Groupe symétrique (fin) ; 2. Applications multilinéaires (début).\\
% \bu\ Exercices : feuille n° 23, ex. 13.\vspace{.4cm}\\
% 
% \noindent\textbf{\bf Jeudi 19 mai 2017 } \\
% $\bullet$\ Distribution : DM n° 20 (à rendre le 26 mai).\\
% \bu\ Cours : \bf Chapitre XXIII \rm : Déterminants : 1 - Groupe symétrique (début).\\
% \bu\ Exercices : feuille n° 23, ex. 11.\\
% $\bullet$\ À faire pour vendredi 20/05 : feuille n° 23, ex. 13.\vspace{.4cm}\\
% 
% \noindent\textbf{\bf Mercredi 18 mai 2017 } \\
% \bu\ Cours : 8 - Matrices par blocs.\\
% \bu\ Exercices : feuille n° 23, ex. 10.\\
% $\bullet$\ À faire pour jeudi 19/05 : feuille n° 23, ex. 11.\vspace{.4cm}\\
% 
% \noindent\textbf{Vendredi 13 mai 2017 }\\
% \bu\ Devoir surveillé n° 09.\\
% \bu\ Cours : 7 - Matrices semblables et trace.\\
% \bu\ Exercices : feuille n° 23, ex. 9.\vspace{.4cm}\\
% 
% \noindent\textbf{\bf Jeudi 12 mai 2017 } \\
% \bu\ Cours : 5 - Rang d'une matrice ; 6 - Systèmes linéaires.\\
% \bu\ Exercices : feuille n° 23, ex. 4.\vspace{.4cm}\\
% 
% \noindent\textbf{\bf Mercredi 11 mai 2017 } \\
% \bu\ Cours : 3 - Matrices remarquables (fin) ; 4 - Opérations sur élémentaires sur les matrices.\\
% \bu\ Exercices : feuille n° 23, ex. 7 et 8.\vspace{.4cm}\\
% 
% \noindent\textbf{\bf Lundi 09 mai 2017 } \\
% \bu\ Interrogation n° 19.\\\\
% \bu\ Cours : 3 - Matrices remarquables (début).\\
% \bu\ Exercices : feuille n° 23, ex. 1, 2, 3, 5 et 6.\vspace{.4cm}\\
% 
% \noindent\textbf{\bf Mercredi 04 mai 2017 } \\
% \bu\ Cours : 2 - Matrices, familles de vecteurs et applications linéaires (fin).\\
% $\bullet$\ À faire pour lundi 09/05 : feuille n° 23, ex. 5.\\
% \bu\ Exercices : feuille n° 22, ex. 11 et 12.\vspace{.4cm}\\
% 
% \noindent\textbf{\bf Lundi 02 mai 2017 } \\
% \bu\ Interrogation n° 18.\\
% \bu\ Cours : 2 - Matrices, familles de vecteurs et applications linéaires (suite).\\
% \bu\ Exercices : feuille n° 22, ex. 6 à 10.\vspace{.4cm}\\
% 
% \noindent\textbf{Vendredi 29 avril 2017 }\\
% \bu\ Cours : 2 - Matrices, familles de vecteurs et applications linéaires (début).\vspace{.4cm}\\
%  
% \noindent\textbf{Jeudi 28 avril 2017 }\\
% $\bullet$\ Distribution : DM n° 19 (à rendre le 09 mai).\\
% \bu\ Cours : 2 - Variables aléatoires (fin).\\
% \bf Chapitre XXII \rm : Matrices ; 1 - Structure de $\mcal M_{n,p}(\K)$.\vspace{.4cm}\\
% 
% \noindent\textbf{ Mercredi 27 avril 2017 } \\
% \bu\ Cours : 2 - Variables aléatoires (suite).\\
% \bu\ Exercices : feuille n° 22, ex. 5 et 6 (début).\vspace{.4cm}\\
% 
% \noindent\textbf{\bf Lundi 25 avril 2017 } \\
% \bu\ Cours : 2 - Variables aléatoires (suite).\\
% \bu\ Exercices : feuille n° 22, ex. 1 (fin) et 2 à 4.\vspace{.4cm}\\
% 
% \noindent\textbf{ Vacances de printemps }\vspace{.4cm}\\
% 
% \noindent\textbf{Vendredi 08 avril 2017 }\\
% \bu\ Cours : 2 - Variables aléatoires (suite).\\
% \bu\ Exercices : feuille n° 22, ex. 1 (suite).\vspace{.4cm}\\
% 
% \noindent\textbf{Jeudi 07 avril 2017 }\\
% \bu\ Cours : 2 - Variables aléatoires (début).\\
% \bu\ Exercices : feuille n° 22, ex. 1 (début).\vspace{.4cm}\\
% 
% \noindent\textbf{Mercredi 06 avril 2017 }\\
% \bu\ Cours : 1 - Événements (fin).\vspace{.4cm}\\
% \bu\ Exercices : feuille n° 21, ex. 17 à 20.\vspace{.4cm}\\
% 
% \noindent\textbf{\bf Lundi 04 avril 2017 } \\
% $\bullet$\ Distribution : DM n° 18 (à rendre le 28 avril).\\
% \bu\ Cours : \bf Chapitre XXII \rm : Probabilités : 1 - Événements (début).\\
% \bu\ Exercices : feuille n° 21, ex. 10 à 16.\vspace{.4cm}\\
% 
% \noindent\textbf{Vendredi 01 avril 2017 }\\
% \bu\ Cours : 4 - Formes linéaires et hyperplans.\\
% \bu\ Exercices : feuille n° 21, ex. 5, 7, 8 et 9.\\
% \bu\ Devoir surveillé n° 8.\vspace{.4cm}\\
%  
% \noindent\textbf{ Jeudi 31 mars 2017 } \\
% \bu\ Cours : 3 - Applications linéaires en dimension finie (fin).\\
% \bu\ Exercices : feuille n° 20, ex. 7 et 8, et feuille n° 21, ex. 1 à 4.\\
% $\bullet$\ À faire pour vendredi 01/04 : feuille n° 21, ex. 6.\vspace{.4cm}\\
% % 
% \noindent\textbf{ Mercredi 30 mars 2017 } \\
% \bu\ Cours : 3 - Applications linéaires en dimension finie (début).\\
% \bu\ Exercices : feuille n° 20, ex. 1 à 6.\\
% $\bullet$\ À faire pour jeudi 31/03 : feuille n° 20, ex. 7 et 8.\vspace{.4cm}\\
%  
% \noindent\textbf{Vendredi 25 mars 2017 }\\
% \bu\ Cours : 2 - Sev en dimension finie (fin).\\
% \bu\ Exercices : feuille n° 19, ex. 26 à 30.\vspace{.4cm}\\
%  
% \noindent\textbf{Jeudi 24 mars 2017 }\\
% \bu\ Cours : 2 - Sev en dimension finie (début).\\
% \bu\ Exercices : feuille n° 19, ex. 23 et 24.\vspace{.4cm}\\
%  
% \noindent\textbf{ Mercredi 23 mars 2017 } \\
% $\bullet$\ Distribution : DM n° 17 (à rendre le 31 mars).\\
% \bu\ Cours : Espaces vectoriels de dimension finie  : 1 - Notion de dimension (fin).\\
% \bu\ Exercices : feuille n° 19, ex. 19, 20, 22 et 25.\vspace{.4cm}\\
% 
% \noindent\textbf{\bf Lundi 21 mars 2017 } \\
% \bu\ Interrogation n° 17.\\
% \bu\ Cours : \bf Chapitre XXI \rm : Espaces vectoriels de dimension finie  : 1 - Notion de dimension (début).\\
% \bu\ Exercices : feuille n° 19, ex. 12 à 18, et 21.\\
% $\bullet$\ À faire pour mercredi 23/03 : feuille n° 19, ex. 22 et 25.\vspace{.4cm}\\
%  
% \noindent\textbf{Vendredi 18 mars 2017 }\\
% \bu\ Cours : Dénombrement (fin).\vspace{.4cm}\\
%  
% \noindent\textbf{Jeudi 17 mars 2017 }\\
% \bu\ Cours : Dénombrement : suite.\\
% \bu\ Exercices : feuille n° 19, ex. 9 et 11.\vspace{.4cm}\\
%  
% \noindent\textbf{ Mercredi 16 mars 2017 } \\
% $\bullet$\ Distribution : DM n° 16 (à rendre le 24 mars).\\
% \bu\ Cours : \bf Chapitre XX \rm : Dénombrement : début.\\
% \bu\ Exercices : feuille n° 19, ex. 4, 5, 6 et 8.\vspace{.4cm}\\
%  
% \noindent\textbf{\bf Lundi 14 mars 2017 } \\
% \bu\ Cours : Intégration (fin).\\
% \bu\ Exercices : feuille n° 18, ex. 15 et 18 à 23, et feuille n° 19, ex. 1 à 3.\vspace{.4cm}\\
%  
% \noindent\textbf{Vendredi 11 mars 2017 }\\
% \bu\ Devoir surveillé n° 7.\\
% \bu\ Exercices : feuille n° 18, ex. 13 et 14.\vspace{.4cm}\\
% 
% \noindent\textbf{Jeudi 10 mars 2017 }\\
% \bu\ Cours : Intégration : 2 - Construction de l'intégrale (fin).\\
% \bu\ Exercices : feuille n° 18, ex. 11 et 12.\\
% $\bullet$\ À faire pour vendredi 11/03 : feuille n° 18, ex. 16.\vspace{.4cm}\\
%  
% \noindent\textbf{ Mercredi 09 mars 2017 }\\
% \bu\ Cours : \bf Chapitre XIX : \rm : Intégration : 1 - Continuité uniforme ; 2 - Construction de 
% l'intégrale (début).\\
% \bu\ Exercices : feuille n° 18, ex. 7 à 10.\\
% $\bullet$\ À faire pour jeudi 10/03 : feuille n° 18, ex. 11.\vspace{.4cm}\\
%  
% \noindent\textbf{Lundi 07 mars 2017 }\\
% \bu\ Interrogation n° 16.\\
% \bu\ Cours : Espaces vectoriels : 5 - Familles de vecteurs (fin) ; 6 - Endomorphismes particuliers.\\
% \bu\ Exercices : feuille n° 18, ex. 1 à 6.\\
% $\bullet$\ À faire pour mercredi 09/03 : feuille n° 18, ex. 8.\vspace{.4cm}\\
%  
% \noindent\textbf{\bf Vendredi 04 mars 2017 } \\
% \bu\ Cours : Espaces vectoriels : 5 - Familles de vecteurs (suite).\\
% \bu\ Exercices : feuille n° 17, ex. 32.\vspace{.4cm}\\
%  
% \noindent\textbf{Jeudi 03 mars 2017 }\\
% $\bullet$\ Distribution : DM n° 15 (à rendre le 10 mars).\\
% \bu\ Cours : Espaces vectoriels : 5 - Familles de vecteurs (suite).\\
% \bu\ Exercices : feuille n° 17, ex. 27, 29 et 30.\vspace{.4cm}\\
% 
% \noindent\textbf{ Mercredi 02 mars 2017 } \\
% \bu\ Cours : Espaces vectoriels : 5 - Familles de vecteurs (suite).\\
% \bu\ Exercices : des calculs de DL.\\
% $\bullet$\ À faire pour jeudi 03/03 : feuille n° 17, ex. 27.\\
% $\bullet$\ À faire pour vendredi 04/03 : feuille n° 17, ex. 32.\vspace{.4cm}\\
%  
% \noindent\textbf{\bf Lundi 29 février 2017 } \\
% \bu\ Interrogation n° 15.\\
% \bu\ Cours : Espaces vectoriels : 4 - Applications linéaires (fin) ; 5 - Familles de vecteurs (début).\\
% \bu\ Exercices : des calculs de DL et feuille n° 17, ex. 20, 21, 25 et 26.\vspace{.4cm}\\
% 
% \noindent\textbf{\bf Vacances d'hiver }\\
% 
% \noindent\textbf{Vendredi 12 février 2017 }\\
% \bu\ Cours : Espaces vectoriels : 4 - Applications linéaires (début).\\
% \bu\ Exercices : feuille n° 17, ex. 16 et 19 (début).\vspace{.4cm}\\
% 
% \noindent\textbf{Jeudi 11 février 2017 }\\
% \bu\ Cours : Espaces vectoriels : 3 - Sea.\\
% \bu\ Exercices : des calculs de DL.\vspace{.4cm}\\
%  
% \noindent\textbf{ Mercredi 10 février 2017 } \\
% $\bullet$\ Distribution : DM n° 14 (à rendre le 03 mars).\\
% \bu\ Cours : Espaces vectoriels : 2 - Sev (fin).\\
% \bu\ Exercices : feuille n° 17, ex. 11 à 13.\vspace{.4cm}\\
%  
% \noindent\textbf{\bf Lundi 08 février 2017 } \\
% \bu\ Cours : Espaces vectoriels : 2 - Sev (suite).\\
% \bu\ Exercices : feuille n° 17, ex. 1 (fin), 4 à 10 et 31.\vspace{.4cm}\\
% 
% \noindent\textbf{Vendredi 05 février 2017 }\\
% \bu\ Devoir surveillé n° 6.\\
% \bu\ Cours : Espaces vectoriels : 2 - Sev (début).\\
% \bu\ Exercices : feuille n° 17, ex. 2 et 3, et début de l'exercice 1.\\
% \vspace{.4cm}\\
%  
% \noindent\textbf{Jeudi 04 février 2017 }\\
% \bu\ Cours : \bf Chapitre XVIII \rm : Espaces vectoriels : 1 - Ev.\\ 
% \bu\ Exercices : feuille n° 16, ex. 7.\\
% $\bullet$\ À faire pour vendredi 06/02 : feuille n° 17, ex. 2 et 3.\vspace{.4cm}\\
%  
% \noindent\textbf{ Mercredi 03 février 2017 } \\
% \bu\ Cours : Analyse asymptotique (fin).\\
% \bu\ Exercices : feuille n° 16, ex. 8.\vspace{.4cm}\\
%  
% \noindent\textbf{\bf Lundi 01 février 2017 } \\
% \bu\ Interrogation n° 14.\\
% \bu\ Cours : Analyse asymptotique (suite).\\
% \bu\ Exercices : feuille n° 16, ex. 1 à 6.\vspace{.4cm}\\
%  
% \noindent\textbf{Vendredi 29 janvier 2017 }\\
% \bu\ Cours : Analyse asymptotique (suite).\\
% \bu\ Exercices : feuille n° 15, ex. 17.\vspace{.4cm}\\
%  
% \noindent\textbf{Jeudi 28 janvier 2017 }\\
% \bu\ Cours : Analyse asymptotique (suite).\\
% \bu\ Exercices : feuille n° 15, ex. 12 et 14 à 16.\\
% $\bullet$\ À faire pour vendredi 29/01 : feuille n° 15, ex. 17.\vspace{.4cm}\\
% 
% \noindent\textbf{ Mercredi 27 janvier 2017 } \\
% $\bullet$\ Distribution : DM n° 13 (à rendre le 04 février).\\
% \bu\ Cours : Analyse asymptotique (suite).\\
% \bu\ Exercices : feuille n° 15, ex. 7 (fin) et 13.\\
% $\bullet$\ À faire pour vendredi 29/01 : feuille n° 17, ex. 5.\vspace{.4cm}\\
%  
% \noindent\textbf{ Lundi 25 janvier 2017 } \\
% \bu\ Interrogation n° 13.\\
% \bu\ Cours : Fractions rationnelles (fin).\\
% \bf Chapitre XVII \rm : Analyse asymptotique (début).\\
% \bu\ Exercices : feuille n° 15, ex. 1 et 4 à 10.\\
% $\bullet$\ À faire pour mercredi 28/01 : feuille n° 15, ex. 7 (fin).\vspace{.4cm}\\
%  
% \noindent\textbf{Vendredi 22 janvier 2017 }\\
% \bu\ Cours : Fractions rationnelles (suite).\\
% \bu\ Exercices : feuille n° 15, ex. 4 (suite).\vspace{.4cm}\\
%  
% \noindent\textbf{Jeudi 21 janvier 2017 }\\
% \bf Chapitre XVI \rm : Fractions rationnelles (début).\\
% $\bullet$\ Distribution : DM n° 12 (à rendre le 28 janvier).\\
% \bu\ Exercices : feuille n° 15, ex. 3 et 4 (début).\vspace{.4cm}\\
%  
% \noindent\textbf{ Mercredi 20 janvier 2017 } \\
% $\bullet$\ Distribution : DM n° 11 (à rendre le 29 janvier).\\
% \bu\ Cours : Dérivabilité (fin).\\
% \bu\ Exercices : feuille n° 14, ex. 18.\vspace{.4cm}\\
%  
% \noindent\textbf{ Lundi 18 janvier 2017 } \\
% \bu\ Cours : Dérivabilité (suite).\\
% \bu\ Exercices : feuille n° 14, ex. 8, 9 et 14 à 16, feuille 
% n° 15, ex. 11 (fin).\\
% \bu\ À faire pour mercedi 20 : feuille n° 14, ex. 18.\vspace{.4cm}\\
%  
% \noindent\textbf{Vendredi 15 janvier 2017 }\\
% \bu\ Devoir surveillé n° 5.\\
% \bu\ Cours : \bf Chapitre XV \rm : Dérivabilité (début).\\
% \bu\ Exercices : feuille n° 14, ex. 3, et feuille n° 15, ex. 11 (début).\\
% \bu\ À faire pour lundi 18 : feuille n° 15, ex. 11 (à finir).\vspace{.4cm}\\
%  
% \noindent\textbf{Jeudi 14 janvier 2017 }\\
% \bu\ Cours : Polynômes (fin).\\
% \bu\ Exercices : feuille n° 14, ex. 17, 19 et 13 (début).\\
% \bu\ À faire pour vendredi 15 : feuille n° 14, ex. 3, et 13 (à finir).\vspace{.4cm}\\
% 
% \noindent\textbf{ Mercredi 13 janvier 2017 } \\
% \bu\ Cours : Polynômes (suite).\\
% \bu\ Exercices : feuille n° 14, ex. 20.\vspace{.4cm}\\
%  
% \noindent\textbf{ Lundi 11 janvier 2017 } \\
% \bu\ Interrogation n° 12.\\
% \bu\ Cours : Polynômes (suite).\\
% \bu\ Exercices : feuille n° 14, ex. 1, 2, 4, 7, 10, 11 et 12.\vspace{.4cm}\\
%  
% \noindent\textbf{Vendredi 09 janvier 2017 }\\
% \bu\ Cours : Polynômes (suite).\\
% \bu\ Exercices : feuille n° 13, ex. 3 et 14.\\
% \bu\ À faire pour lundi 11 : feuille n° 14, ex. 4.\vspace{.4cm}\\
%  
% \noindent\textbf{Jeudi 07 janvier 2017 }\\
% $\bullet$\ Distribution : DM n° 11 (à rendre le 14 janvier).\\
% \bu\ Cours : Polynômes (suite).\\
% \bu\ Exercices : feuille n° 7, 8, 10, 11 et 13.\\
% \bu\ À faire pour vendredi 08 : feuille n° 13, ex. 3.\vspace{.4cm}\\
%  
% \noindent\textbf{ Mercredi 06 janvier 2017 } \\
% \bu\ Cours : Polynômes (suite).\\
% \bu\ Exercices : feuille n° 13, ex. 12.\\
% \bu\ À faire pour jeudi 07 : feuille n° 13, ex. 13.\vspace{.4cm}\\
%  
% \noindent\textbf{ Lundi 04 janvier 2017 } \\
% \bu\ Interrogation n° 11.\\
% \bu\ Cours : Polynômes (suite).\\
% \bu\ Exercices : feuille n° 13, ex. 1, 2, 4, 5, 6 et 9.\vspace{.4cm}\\
%   
% \noindent\textbf{\bf Vacances de Noël }\\
%  
% \noindent\textbf{Vendredi 18 décembre 2016 }\\
% $\bullet$\ Distribution : DM n° 10 (à rendre le 07 janvier).\\
% $\bullet$\ \bf Chapitre XIV \rm : Polynômes (début).\\
% \bu\ Exercices : feuille n° 12, ex. 6 et 7.\vspace{.4cm}\\
%  
% \noindent\textbf{Jeudi 17 décembre 2016 }\\
% \bu\ Cours : Continuité (fin).\\
% \bu\ Exercices : feuille n° 12, ex. 1 (fin), 2, 3 et 5.\vspace{.4cm}\\
%  
% \noindent\textbf{\bf Mercredi 16 décembre 2016 } \\
% \bu\ Cours : Continuité (suite).\\
% \bu\ Exercices : feuille n° 12, ex. 1 (début).\\
% \bu\ À faire pour vendredi 18 : feuille n° 12, ex. 1 (à finir).\vspace{.4cm}\\
%  
% \noindent\textbf{ Lundi 14 décembre 2016 } \\
% \bu\ Interrogation n° 10.\\
% \bu\ Cours : Limites d'une fonction (fin).\\
% $\bullet$\ \bf Chapitre XIII \rm : Continuité (début).\\
% \bu\ Exercices : feuille n° 11, ex. 4 à 12.\vspace{.4cm}\\
%  
% \noindent\textbf{Vendredi 11 décembre 2016 }\\
% \bu\ Cours : Limites d'une fonction (suite).\\
% \bu\ Exercices : feuille n° 11, ex. 2.\\
% \bu\ À faire pour lundi 14 : feuille n° 11, ex. 3.\vspace{.4cm}\\
%  
% \noindent\textbf{Jeudi 10 décembre 2016 }\\
% \bu\ Cours : Groupes, anneaux, corps (fin).\\
% \bf Chapitre XII \rm : Limites d'une fonction (début).\\
% $\bullet$\ Distribution : DM n° 9 (à rendre le 17 décembre).\\
% \bu\ Exercices : feuille n° 11, ex. 1 et 2 (début).\\
% \bu\ À faire pour vendredi 11 : feuille n° 11, ex. 2 (à finir).\vspace{.4cm}\\
%  
% \noindent\textbf{\bf Mercredi 09 décembre 2016 } \\
% \bu\ Cours : Groupes, anneaux, corps (suite).\\
% \bu\ Exercices : feuille n° 10, ex. 12.\\
% \bu\ À faire pour jeudi 11 : feuille n° 11, ex. 1.\vspace{.4cm}\\
%  
% \noindent\textbf{ Lundi 07 décembre 2016 } \\
% \bf Chapitre XI \rm : Groupes, anneaux, corps (début).\\
% \bu\ Exercices : feuille n° 10, ex. 4 et 11 à 17.\\
% \bu\ À faire pour mercedi 10 : feuille n° 10, ex. 12 (à finir)\vspace{.4cm}\\
%  
% \noindent\textbf{Vendredi 04 décembre 2016 }\\
% \bu\ Devoir surveillé n° 4.\\
% $\bullet$\ Cours : 7 : suites complexes ; 8 - Premières séries numériques.\\
% \bu\ Exercices : feuille n° 10, ex. 8 à 10.\vspace{.4cm}\\
%  
% \noindent\textbf{Jeudi 03 décembre 2016 }\\
% \bu\ Cours : 6 - Suites récurrentes.\\
% \bu\ Exercices : feuille n° 10, ex. 6, 3 et 4 (début).\\
% \bu\ À faire pour vendredi 04 : feuille n° 10, ex. 4 (à finir).\vspace{.4cm}\\
%  
% \noindent\textbf{\bf Mercredi 02 décembre 2016 } \\
% \bu\ Cours : 4 et 5 - Suites particulières.\\
% \bu\ Exercices : feuille n° 10, ex. 5.\\
% \bu\ À faire pour jeudi 03 : feuille n° 10, ex. 6.\vspace{.4cm}\\
%  
% \noindent\textbf{ Lundi 30 novembre 2016 } \\
% \bu\ Interrogation n° 9.\\
% \bu\ Cours : 3 - Résultats de convergence.\\
% \bu\ Exercices : feuille n° 10, ex. 1, 2 et 7.\\
% \bu\ À faire pour mercredi 02 : feuille n° 10, ex. 5.\vspace{.4cm}\\
%  
% \noindent\textbf{Vendredi 27 novembre 2016 }\\
% \bu\ Cours : 2.4 - Inégalités.\\
% \bu\ Exercices : feuille n° 9, ex. 16 et 17.\vspace{.4cm}\\
%  
% \noindent\textbf{Jeudi 26 novembre 2016 }\\
% $\bullet$\ Distribution : DM n° 8 (à rendre le 03 décembre).\\
% \bu\ Cours : 2.2 - Opérations ; 2.3 : Suites extraites.\\
% \bu\ Exercices : feuille n° 9, ex. 12 à 15.\vspace{.4cm}\\
%   
% \noindent\textbf{\bf Mercredi 25 novembre 2016 } \\
% $\bullet$\ Cours : \bf Chapitre X :\rm Suites réelles et complexes : 1 - Vocabulaire ; 2 - Limite 
% d'une suite réelle (suite).\\
% \bu\ Exercices : feuille n° 9, ex. 11.\\
% \bu\ À faire pour jeudi 26 : feuille n° 9, ex. 13 et 14.
% \vspace{.4cm}\\
%   
% \noindent\textbf{ Lundi 23 novembre 2016 } \\
% \bu\ Interrogation n° 8.\\
% $\bullet$\ Cours : \bf Chapitre X :\rm Suites réelles et complexes : 1 - Vocabulaire ; 2 - Limite 
% d'une suite réelle (début).\\
% \bu\ Exercices : feuille n° 8, ex. 4 et feuille n° 9, ex. 1 à 
% 10.\vspace{.4cm}\\
%   
% \noindent\textbf{Vendredi 20 novembre 2016 }\\
% \bu\ Cours : 3 - Nombres premiers.\\
% \bu\ Exercices : feuille n° 8, ex. 11 et 4 (début).\\
% \bu\ À faire pour lundi 23 : feuille n° 8, ex. 4.\vspace{.4cm}\\
%   
% \noindent\textbf{Jeudi 19 novembre 2016 }\\
% \bu\ Cours : 2 - PPCM.\\
% \bu\ Exercices : feuille n° 8, ex. 8 à 10.\\
% \bu\ À faire pour vendredi 20 novembre : feuille n° 8, ex. 11.\vspace{.4cm}\\
%   
% \noindent\textbf{\bf Mercredi 18 novembre 2016 } \\
% $\bullet$\ Distribution : DM n° 7 (à rendre le 26 novembre).\\
% \bu\ Cours : 2 - PGCD.\\
% \bu\ Exercices : feuille n° 8, ex. 6 (suite).\vspace{.4cm}\\
%   
% \noindent\textbf{\bf Lundi 16 novembre 2016 }\\
% $\bullet$\ Cours : \bf Chapitre IX \rm : Arithmétique dans \Z\ : 1 - 
% Divisibilité.\\
% \bu\ Exercices : feuille n° 7, ex. 9 et feuille n° 8, ex. 1, 2, 3, 5 et 6 (début).\\
% \bu\ À faire pour mercredi 18 novembre : feuille n° 8, ex. 6 (fin).\vspace{.4cm}\\
%  
% \noindent\textbf{Vendredi 13 novembre 2016 }\\
% \bu\ Devoir surveillé n° 3.\\
% \bu\ Cours : 5 - La relation d'ordre naturelle sur \R\ (fin).\\
% \bu\ Exercices : feuille n° 7, ex. 8 et 10.\\
% \bu\ À faire pour lundi 16 novembre : feuille n° 7, ex. 9.\vspace{.4cm}\\
% 
% \noindent\textbf{Jeudi 12 novembre 2016 }\\
% \bu\ Cours : 4 - Majorants, minorants et 
% compagnie (fin ; 5 - La relation d'ordre naturelle sur \N\ ; 6 - La relation d'ordre 
% naturelle sur \R\ (début).\\
% \bu\ Exercices : feuille n° 7, ex. 6 et 8 (début).\\
% \bu\ À faire pour vendredi 13 novembre : feuille n° 7, ex. 8, qu. 2 à 4.\vspace{.4cm}\\
% 
% \noindent\textbf{ Lundi 09 novembre 2016 } \\
% $\bullet$\ Interrogation surprise n° 7.\\
% $\bullet$\ Cours : \bf Chapitre VIII \rm : Relations d'ordre : 1 - Relations 
% binaires ; 2 - Relations d'équivalence ; 3 - Relations d'ordre ; 4 - Majorants, minorants et 
% compagnie (début).\\
% \bu\ Exercices : feuille n° 7, ex. 4, 5 et 7.\vspace{.4cm}\\
%  
% \noindent\textbf{Vendredi 06 novembre 2016 }\\
% $\bullet$\ Cours : 5 - Un peu de physique.
% \bu\ Exercices : feuille n° 7, ex. 3 (fin) et 4 (qu. 4, 5, 6 et 8).\\
% \bu\ À faire pour lundi 09 novembre : feuille n° 7, ex. 7, qu. 1 à 4.\vspace{.4cm}\\
%  
% \noindent\textbf{Jeudi 05 novembre 2016 }\\
% $\bullet$\ Distribution : DM n° 6 (à rendre le 12 novembre).\\
% $\bullet$\ Cours : 4 - Équations différentielles linéaires du second ordre.\\
% \bu\ Exercices : feuille n° 7, ex. 3 (début).\vspace{.4cm}\\
%  
% \noindent\textbf{\bf Mercredi 04 novembre 2016 } \\
% $\bullet$\ Cours : 3.2 - Équations différentielles du premier ordre avec 
% second membre.\\
% \bu\ Exercices : feuille n° 7, ex. 2.\\
% \bu\ À faire pour jeudi 05 novembre : feuille n° feuille n° 7, 
% ex. 4, qu. 4, 5, 6 et 8.\vspace{.4cm}\\
%   
% \noindent\textbf{ Lundi 02 novembre 2016 } \\
% $\bullet$\ Interrogation surprise n° 6.\\
% $\bullet$\ Cours : 2 - Généralités (fin) ; 3.1 - Équations différentielles 
% homogènes du premier ordre.\\
% \bu\ Exercices : feuille n° 6, ex. 08 et 14, et feuille n° 7, ex.
% 1.\vspace{.4cm}\\
%  
% \noindent\textbf{ Vacances de novembre }\vspace{.4cm}\\
%  
% \noindent\textbf{Vendredi 16 octobre 2016 }\\
% $\bullet$\ Cours : 1 - Résultats d'analyse relatifs aux fonctions à valeurs
% complexes d'une variable réelle, et intégration (fin) ; 2 - Généralités (début).\\
% $\bullet$\ Exercices : feuille n° 6, ex. 9 et 13.\\
% \bu\ À faire pour lundi 02 novembre : feuille n° 6, ex. 8 et 14, et feuille n° 7, 
% ex. 1.\vspace{.4cm}\\
%  
% \noindent\textbf{\bf Jeudi 15 octobre 2016 } \\
% $\bullet$\ Distribution : DM n° 5 (à rendre le 05 novembre).\\
% $\bullet$\ Cours : 1 - Résultats d'analyse relatifs aux fonctions à valeurs
% complexes d'une variable réelle, et intégration (suite).\\
% $\bullet$\ Exercices : feuille n° 6, ex. 6, 7, 10, 11, 12, 15 et 16.
% \bu\ À faire pour vendredi 16 : feuille n° 6, ex. 13.\vspace{.4cm}\\
% 
% \noindent\textbf{\bf Mercredi 14 octobre 2016 } \\
% $\bullet$\ Cours : \bf Chapitre VII \rm : Équations différentielles linéaires : 1 - Résultats 
% d'analyse relatifs aux fonctions à valeurs complexes d'une variable réelle, et intégration 
% (début).\\
% $\bullet$\ Exercices : feuille n° 6, ex. 2 et 5.\\
% \bu\ À faire pour jeudi 15 : feuille n° 6, ex. 6 et 7.\vspace{.4cm}\\
% 
% \noindent\textbf{Lundi 12 octobre 2016 }\\
% $\bullet$\ Interrogation surprise n° 5.\\
% $\bullet$\ Cours :  6 - Fonctions trigonométriques inverses (fin) ; 7 -
% Fonctions hyperboliques.\\
% $\bullet$\ Exercices : feuille n° 5, ex. 5 (fin), 7 et 8, et feuille n° 6, ex.
% 1 et 4.\vspace{.4cm}\\
%  
% \noindent\textbf{Vendredi 09 octobre 2016 }\\
% \bu\ Devoir surveillé n° 2.\\
% $\bullet$\ Cours :  5 - Fonctions exponentielle, logarithme et 
% exponentielles de base $a$ (fin) ; 6 - Fonctions trigonométriques inverses (début).\\
% $\bullet$\ Exercices : feuille n° 5, ex. 5 (début).\vspace{.4cm}\\
%  
% \noindent\textbf{\bf Jeudi 08 octobre 2016 } \\
% $\bullet$\ Cours : 3 - Valeur absolue ; 4 - Puissances entières, fonctions 
% polynomiales et rationnelles ; 5 - Fonctions exponentielle, logarithme et 
% exponentielles de base $a$ (début).\\
% $\bullet$\ Exercices : feuille n° 5, ex. 6, 9, 10 et 12.\vspace{.4cm}\\
% 
% \noindent\textbf{\bf Mercredi 07 octobre 2016 } \\
% $\bullet$\ Cours : 2 - Théorèmes d'analyse admis.\\
% \bu\ Exercices : à faire pour jeudi 08 : ex. 2.0.9 du chapitre VI, et feuille n° 6, ex. 
% 1.\vspace{.4cm}\\
%   
% \noindent\textbf{Lundi 05 octobre 2016 }\\
% $\bullet$\ Interrogation surprise n° 4.\\
% \bu\ Cours : 4 - Bijectivité (début) ; 5 - Images directe et réciproque.\\
% \es\ \bf Chapitre VI \rm : Fonctions usuelles : 1 - Vocabulaire usuel (début).\\
% $\bullet$\ Exercices : feuille n° 4, ex. 8, et feuille n° 5, ex. 1, 3 et 4.\vspace{.4cm}\\
%   
% \noindent\textbf{Vendredi 02 octobre 2016 }(2 +4 \\
% \bu\ Devoir surveillé n° 2.\vspace{.4cm}\\
% \bu\ Cours : 4 - Bijectivité (début).\\
% $\bullet$\ Exercices : feuille n° 4, ex. 6 et 7.\\
% \bu\ À faire pour lundi 05 : feuille n° 4, ex. 8.\vspace{.4cm}\\
% 
% \noindent\textbf{Jeudi 01 octobre 2016 }\\
% $\bullet$\ Distribution : DM n° 4 (à rendre le 08 octobre).\\
% \bu\ Cours : 4 - Injectivité, surjectivité.\\
% $\bullet$\ Exercices : feuille n° 4, ex. 1 à 5.\\
% \bu\ À faire pour vendredi 02 : feuille n° 4, ex. 6.\vspace{.4cm}\\
%  
% \noindent\textbf{\bf Mercredi 30 septembre 2016 } \\
% $\bullet$\ Cours : Théorie des ensembles (fin).\\
% \bf Chapitre V \rm : Notion d'application. 1 - Vocabulaire ;
% 2 - Restriction et prolongement ; 3 - Composition.\vspace{.4cm}\\
%  
% \noindent\textbf{Lundi 28 septembre 2016 }\\
% $\bullet$\ Interrogation surprise n° 3.\\
% $\bullet$\ Exercices : feuille n° 3, ex. 10, 16, 18, 19 et 20.\vspace{.4cm}\\
%  

\noindent\textbf{Vendredi 30 septembre 2016 }\\
$\bullet$\ Cours : \bf Chapitre V \rm : Notion d'application. 1 - Vocabulaire ; 2 - Restriction et prolongement.\\
$\bullet$\ Exercices : feuille n° 3, ex. 12, 13 et 14 (fin), .\\
$\bullet$\ À faire pour lundi 03 : feuille n° 3, ex. 18 à 21.\vspace{.4cm}\\

\noindent\textbf{Jeudi 29 septembre 2016 }\\
$\bullet$\ Cours : Théorie des ensembles (début).\\
$\bullet$\ Exercices : feuille n° 3, ex. 8, 11, 13 et 14 (début).\\
$\bullet$\ À faire pour vendredi 30 : feuille n° 3, ex. 14 (fin).\vspace{.4cm}\\

\noindent\textbf{\bf Mardi 27 septembre 2016 }\\
$\bullet$\ Cours : 5 - Systèmes linéaires et pivot de Gauss (fin).\\
$\bullet$\ Exercices : feuille n° 3, ex. 2, 4, 7 et 9.\\
$\bullet$\ À faire pour jeudi 29 : feuille n° 3, ex. 8.\vspace{.4cm}\\

\noindent\textbf{Lundi 26 septembre 2016 }\\
$\bullet$\ Interrogation surprise n° 2.\\
$\bullet$\ Cours : 5 - Systèmes linéaires et pivot de Gauss (suite).\\
$\bullet$\ Exercices : feuille n° 3, ex. 1, 3, 5 et 6.\\
$\bullet$\ À faire pour mardi 27 : feuille n° 3, ex. 4 et 7.\vspace{.4cm}\\

\noindent\textbf{Vendredi 23 septembre 2016 }\\
$\bullet$\ Cours : 4 - Calcul matriciel élémentaire (fin) ; 5 - Systèmes linéaires et pivot de Gauss (début).\\
$\bullet$\ Exercices : feuille n° 2, ex. 10 et 12.\vspace{.4cm}\\

\noindent\textbf{Jeudi 22 septembre 2016 }\\
$\bullet$\ Cours : 3 - Quelques formules (fin) ; 4 - Calcul matriciel
élémentaire (début).\\
$\bullet$\ Exercices : feuille n° 2, ex. 9 à 11.\\
$\bullet$\ À faire pour vendredi 23 : montrer que les puissances d'une matrice carrée commutent.\vspace{.4cm}\\

\noindent\textbf{\bf Mardi 20 septembre 2016 }\\
$\bullet$\ Cours : 2 - Produits ; 3 - Quelques formules (début).\vspace{.4cm}\\

\noindent\textbf{Lundi 19 septembre 2016 }\\
$\bullet$\ Cours : Quelques fondamentaux (fin).\\
\bf Chapitre III \rm : Un peu de calcul. 1 - Sommes.\\
$\bullet$\ Exercices : feuille n° 2, ex. 1 à 9.\\
$\bullet$\ À faire pour mardi 20 : récurrences erronées et feuille n° 2, ex. 11.\vspace{.4cm}\\

\noindent\textbf{Vendredi 16 septembre 2016 }\\
\bu\ Devoir surveillé n° 01.\\
$\bullet$\ Cours : Quelques fondamentaux (suite).\vspace{.4cm}\\

\noindent\textbf{Jeudi 15 septembre 2016 }\\
$\bullet$\ Cours : Quelques fondamentaux (suite).\\
$\bullet$\ Exercices : feuille n° 1, ex. 14, 16, 17, 19, 20, 21, 22 et 25 (début).\\
$\bullet$\ À faire pour vendredi 16 : feuille n° 1, ex. 25 (fin).\vspace{.4cm}\\
 
\noindent\textbf{\bf Mardi 13 septembre 2016 }\\
$\bullet$\ Cours : 5 - Nombres complexes et géométrie plane (fin).\\
$\bullet$\ Cours : \bf Chapitre II \rm : Quelques fondamentaux (début).\\
$\bullet$\ Exercices : feuille n° 1, ex. 9 (fin), 10, 12, 13 et 18 (début).\\
$\bullet$\ À faire pour jeudi 15 : feuille n° 01, ex. 14 et 16.\vspace{.4cm}\\

\noindent\textbf{\bf Lundi 12 septembre 2016 }\\
$\bullet$\ Interrogation surprise n° 1.\\
$\bullet$\ Cours : 5 - Nombres complexes et géométrie plane (suite).\\
$\bullet$\ Exercices : feuille n° 1, ex. 11 (fin), 4, 7, et 9 (début).\\
$\bullet$\ À faire pour mardi 13 : feuille n° 1, ex. 9 à finir.\vspace{.4cm}\\
 
\noindent\textbf{Vendredi 09 septembre 2016 }\\
$\bullet$\ Distribution du DM n° 2 (à rendre le 15 septembre).\\
$\bullet$\ Cours : 5 - Nombres complexes et géométrie plane (début).\\
$\bullet$\ Exercices : feuille n° 2, ex. 15.\vspace{.4cm}\\

\noindent\textbf{\bf Jeudi 08 septembre 2016 }\\
$\bullet$\ Cours : 2 - Le groupe \U\ des nombres complexes de 
module 1 (fin) ; 3 - Équations du second degré ; 4 - L'exponentielle complexe.\\
$\bullet$\ Exercices : feuille n° 2, ex. 5, 6, 8 et 11 (début).\\
$\bullet$\ À faire pour vendredi 09 : feuille n° 1, ex. 15.\vspace{.4cm}\\
  
\noindent\textbf{\bf Mardi 06 septembre 2016 }\\
$\bullet$\ Cours : 2 - Le groupe \textbf{U} des nombres complexes de 
module 1 (suite).\\
$\bullet$\ Exercices : feuille n° 01, ex. 1 (fin), et 3 (fin).\\
$\bullet$\ À faire pour jeudi 08 : feuille n° 01, ex. 5.\vspace{.4cm}\\

\noindent\textbf{\bf Lundi 05 septembre 2016 }\\
$\bullet$\ Cours : 1 - Construction de \C\ (fin) ; 2 - Le groupe \textbf{U} des nombres complexes de 
module 1 (début).\\
$\bullet$\ Exercices : feuille n° 01, ex. 1 (début), 2 et 3 (début).\\
$\bullet$\ À faire pour mardi 06 : finir l'ex. 1.\vspace{.4cm}\\

\noindent\textbf{\bf Jeudi 01 septembre 2016 }\\
Journée de rentrée ; distribution des feuilles de TD, des formulaires, des
chapitres I et II, et du DM n° 01 (à rendre le 09 septembre).\\
$\bullet$\ Cours : \bf Chapitre I \rm : Les nombres complexes. 1 - Construction
de \C\ (début).\vspace{.4cm}\\


\label{end}
\end{document}


