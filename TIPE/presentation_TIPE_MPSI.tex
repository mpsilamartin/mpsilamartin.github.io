\documentclass{beamer}
\usepackage[french]{babel}
\RequirePackage[T1]{fontenc}
\RequirePackage[utf8x]{inputenc}
\RequirePackage{lmodern}

\usetheme{Singapore}

\title{TIPE en MPSI}
\author{S. Kleim, C. Kumar, M. Solnon, S. Zannad}
\institute{Lycée La Martinière Monplaisir, Lyon}
\date{2020-2021}

\begin{document}

\begin{frame}
\titlepage
\end{frame}
\section{En MPSI}
\begin{frame}{En MPSI}
\begin{itemize}
\item Thème de l'année : « Santé, prévention » (à confirmer). 
\item Au second semestre, le jeudi, de 13 h 35 à 15 h 25.
\item MPSI 1 et MPSI 2 travaillent ensemble. 
\item Matières : informatique, physique, maths. 
\item Soutenance en fin d'année. 
\end{itemize}

\end{frame}

\section{En spé}
\begin{frame}{En spé}
\begin{itemize}
\item Même thème qu'en sup. 
\item Vous pouvez poursuivre votre sujet de sup (sans obligation). 
\item MP/MP* travaillent ensemble, matières : informatique, physique, maths.
\item Matières en PSI/PSI* : physique, SI. 
\item Premiers documents à téléverser en janvier. 
\end{itemize}
\end{frame}

\begin{frame}{Aux concours}
    \begin{itemize}
        \item Une épreuve pour toutes les écoles (sauf X/ENS). 
        \item Format de l'épreuve : 15 minutes de présentation, 15 minutes de questions. 
        \item Coefficients (filière MP, épreuves orales). 
            \begin{itemize}
                \item CCINP : 8 / 40
                \item Centrale/Supélec : 11 / 100
                \item Mines/Ponts : 6 / 40
            \end{itemize}
    \end{itemize}
    Tous les renseignements sont sur 
    \begin{center}
        \url{https://www.scei-concours.fr/tipe.php}
    \end{center}
\end{frame}

\section{Conseils}
\begin{frame}{Quelques conseils}
    \begin{itemize}
        \item Travaillez en groupes de 2 ou 3.
        \item Formez votre groupe avant le début des TIPE, essayez de trouver une idée de thème qui vous fait envie. 
        \item Pour trouver un sujet précis, partez d'une question ou d'un problème simple et concret. 
        \item Le TIPE doit faire intervenir plusieurs matières et ne doit surtout pas se limiter à l'étude d'une théorie. 
        \item Tenez un cahier de bord (papier ou électronique). Prenez les références précises d'absolument tous les documents étudiés (livres, articles, pages web). 
        \item Sauvegardez systématiquement vos documents sur un support sûr et sécurisé (pas une clef USB). 
    \end{itemize}
\end{frame}

\section{Documents}
\begin{frame}{Documents à rendre}
    \begin{itemize}
        \item Pas de compte-rendu à taper. 
        \item Une fiche de suivi à remplir à intervalles réguliers. 
        \item Point important : la bibliographie. 
        \item À rendre en fin d'année (à confirmer) : bibliographie commentée et DOT. 
        \item En MPSI : un document à rendre par groupe. 
        \item Référence importante à consulter : 
    \end{itemize}

    \begin{center}
        \url{https://www.scei-concours.fr/pdf/AttendusPedagogiques_2021_TIPE.pdf}
    \end{center}
\end{frame}


\begin{frame}{Présentation}
    \begin{itemize}
        \item Format PDF uniquement. 
        \item Prévoir 1 transparent toutes les 1-2 minutes. 
        \item Transparent schématique : pas de longues phrases, pas de pavé de texte. 
        \item Maximum 8-9 lignes par transparent. 
        \item Pas de code informatique sur les transparents (éventuellement des algorithmes en pseudo-code pour les TIPE d'informatique). 
        \item Vous devez réaliser vos propres schémas. 
        \item Présentation personnelle. 
    \end{itemize}

\end{frame}

\end{document}
